\documentclass[12pt]{article}
\usepackage{listings}
\usepackage[utf8]{inputenc}
\usepackage[T1]{fontenc}
\usepackage[polish]{babel}
\usepackage{url}
\title{Zastosowanie algorytmów ewolucyjnych do zagadnień
  kombinatorycznych na przykładzie problemu szeregowania zadań - sprawozdanie}
\author{Michał Karpiński, Maciej Pacut}
\date{Grudzień 2011}
\begin{document}
  \maketitle
\section{Wstęp}
Problem szeregowania zadań ma kilka różnych wariantów. W niniejszej pracy zajmujemy się problemem typu {\em Flow Shop}.
Problem ten jest problemem planowania produkcji, w którym $n$ zadań musi zostać wykonanych na $m$ maszynach (jeden po drugim).
Danymi wejściowymi są $n$, $m$ oraz macierz $T$, gdzie $T[i][j]$ jest czasem wykonania zadania $i$ na maszynie $j$ $(i = 1..n, j = 1..m )$.
Czasy są nieujemne, stałe i nie zmianiają się podczas pracy maszyn. Problemem jest zminimalizowanie czasu 
pomiędzy rozpoczęciem wykonywania pierwszego zadania na pierwszej maszynie a zakończeniem wykonywania ostatniego zadania na ostatniej maszynie.
Dla naszego problemu przyjmujemy nastepujące założenia:

\begin{itemize}
  \item Każde zadanie musi być wykonane co najwyżej raz na maszynach $1, 2 ... m$ (w tej kolejności)
  \item Każda maszyna w danym momencie pracuje tylko nad jednym zadaniem
  \item Każde zadanie w danym momencie jest wykonywane na co najwyżej jednej maszynie
  \item Wykonywanie zadania nigdy nie jest zakłócone (przerwane)
  \item Czas pomiędzy przekazaniem zadania z jednej maszyny na drugą jest zerowy
\end{itemize}

Tak zdefiniowany problem jest NP-trudny. W niniejszej pracy prezentujemy algorytm ewolucyjny poszukujący przybliżonego rozwiązania.

\section{Schemat algorytmu ewolucyjnego}

Do rozwiązania problemu {\em Flow Shop} zastosowaliśmy algorytm
genetyczny SGA wykonujący się wg następującego schematu:

\begin{verbatim}
Losuj populację początkową
while(NOT Warunek Zakończenia)
{
  Krzyżowanie w obrębie populacji
  Mutacja osobników w populacji
  Zastąpienie populacji nową
}
\end{verbatim}

Rodzice dla opratorów krzyżowania są wybierani z prawdopodobieństwem
proporcjonalnym do wartości funkcji przystosowania kandydata.

Długość wymienianego łańcucha jest stałą i wynosi $\frac{1}{3}$ długości
permutacji. Losowane jest miejsce rozpoczęcia wymienianego łańcucha.

Mutacja jest przeprowadzana z prawdopodobieństwem 0.3.

\section{Operatory ewolucyjne}

Zostały wykorzystane operatory krzyżowania permutacji PMX, CX, OX.

\begin{itemize}
  \item {\em Partially-Matched Crossover (PMX)} - 
  \item {\em Cycle Crossover (CX)} - ustalane są cykle między rodziecem pierwszym a drugim, następnie cykle są numerowane. Potomkowie są tworzeni poprzez
        kopiowanie genów należących do cykli nieparzystych (w sensie numeracji) i zamianie genów leżących na pozycjach należących do cykli 
        parzystych na geny drugiego rodzica (na tej samej pozycji)
  \item {\em Order Crossover Operator (OX)} - spójny blok genów z pierwszego rodzica jest mapowany na drugiego rodzica (w tym samym miejscu). Reszta genów jest
        przepisywana kolejno zaczynając od miejsca zakończenia mapowania, pomijając geny występujące w mapowaniu.
\end{itemize}

\noindent Operator mutacji aplikuje losową transpozycję do mutowanej permutacji.

\section{Funkcja celu}

Funkcją celu, czyli wartością, do której będzie dążył algorytm jest czas
pomiędzy rozpoczęciem wykonywania pierwszego zadania na pierwszej maszynie a
zakończeniem ostatniego zadania na ostatniej maszynie. To kryterium optymalizacyjne
nosi nazwę {\em makespan}. Będziemy posługiwali się następującymi oznaczeniami:

\begin{itemize}
  \item Dane rozwiązanie (permutacja $n$-elementowa): $\pi$
  \item Czas obróbki $i$-tego zadania na $j$-tej maszynie: $p_{\pi_i,j}$
  \item Czas zakonczenia $i$-tego zadania na $j$-tej masz: $C(\pi_i,j)$
  \item Kryterium optymalizacyjne: $C_{max}$: $C(\pi_n,m)$
\end{itemize}

Wartość funkcji $C_{max}$ wyraża się wzorem rekurencyjnym:

\begin{itemize}
  \item $C(\pi_1,1) = p_{\pi_1,1}$
  \item $C(\pi_i,1) = C(\pi_{i-1},1) + p_{\pi_i},1$
  \item $C(\pi_1,j) = C(\pi_1,j-1) + p_pi_1,j$
  \item $C(\pi_i,j) = max\{C(\pi_i,j-1),C(\pi_{i-1},j)\} + p_{\pi_i},j$
\end{itemize}

Warto zauważyć, że do obliczenia $C_{max}$ można zastosować algorytm dynamiczny. 

\section{Przeprowadzone testy}

\subsection{Dane testowe}

W celu ustalenia sprawności algorytmu SGA wykorzystane zostały zestawy testów prof. Erica Tailarda
(\url{http://mistic.heig-vd.ch/taillard/problemes.dir/ordonnancement.dir/ordonnancement.html})
, dla których znane są optymalne rozwiązania.

\begin{figure}
  \centering
    \begin{tabular}{ | c | c | c | }
      \hline                       
      Nazwa & N & M \\ \hline
      ta001-010 & 20 & 5 \\
      ta011-020 & 20 & 10 \\
      ta021-030 & 20 & 20 \\
      ta031-040 & 50 & 5 \\
      ta041-050 & 50 & 10 \\
      ta051-060 & 50 & 20 \\
      ta061-070 & 100 & 5 \\
      ta071-080 & 100 & 10 \\
      ta081-090 & 100 & 20 \\
      ta091-100 & 200 & 10 \\
      \hline
\end{tabular}
 \caption{Zestawy testowe, rozmiar danych}
  \label{tab:tests}
\end{figure}

W tabeli \ref{tab:tests} znajduje się uproszczony opis stu zestawów testowych użytych
podczas wykonywania eksperymentów ($N$ - ilość zadań, $M$ - ilość maszyn ). Przeprowadzono testy na różnych wielkościach populacji i dla różnych warunków stopu
(procentowa odległość od optimum).

\subsection{Wariancja}

Zastosowaną metryką w przestrzeni permutacji jest minimalna liczba
transpozycji potrzebnych do przekształcenia jednej permutacji w drugą.

Przedstawiono wykresy wariancji populacji w tej metryce. Testy
pokazują zbieżność populacji w kolejnych iteracjach, nierzadko do
momentu, gdy cała populacja jest kopiami tego samego osobnika.

\subsection{Porównanie z losowym przeszukiwaniem przestrzeni}

Przeprowadzono testy porównujące algorytm genetyczny z losowym
przeszukiwaniem przestrzeni permutacji. Algorytm przeszukujący losowo
przestrzeń wykonywał taką samą liczbę ewaluacji jak algorytm
genetyczny. Otrzymano znacząco lepsze wyniki na korzyść algorytmu genetycznego.

\subsection{Jakość algorytmów krzyżowania}

Przeprowadzono testy sprawdzające, jak często wykorzystane algorytmy
krzyżowania jako wynik otrzymują osobniki lepsze od rodziców. Zostały
zaobserwowane różnice w działaniu operatorów zależnych od wyboru
wymienianego łańcucha (PMX, CX) w stosunku do operatora niezależnego
od tego wyboru (OX). Różnica zależy od wyboru długości wymienianego łańcucha.

\end{document}
