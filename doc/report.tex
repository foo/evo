\documentclass[12pt]{article}
\usepackage[utf8]{inputenc}
\usepackage[T1]{fontenc}
\usepackage[polish]{babel}
\title{Zastosowanie algorytmów ewolucyjnych do zagadnień
  kombinatorycznych na przykładzie problemu szeregowania zadań - sprawozdanie}
\author{Michał Karpiński, Maciej Pacut}
\date{Grudzień 2011}
\begin{document}
  \maketitle
\section{Wstęp}
Sformułowanie problemu. Funkcja celu i sposób jej obliczania.
\section{Schemat algorytmu ewolucyjnego}
SGA. Funkcja przystosowania. Wybór osobników do następnej populacji.
Warunki zakończenia. Prawdopodobieństwo mutacji. Wielkość populacji.
\section{Operatory krzyżowania}
PMX, CX, OX. Wspomnienie o stałej długości s,r.
\section{Operator mutacji}
Cały wielki rodział o jednej transpozycji!
\section{Przeprowadzone testy}
\subsection{Wariancja}
O ``odległości edycyjnej''.
\subsection{Test szybkości zbieżności do optimum}
W porównaniu do losowego przeszukiwania przestrzeni
\subsection{Które operatory krzyżowania psują}
\end{document}